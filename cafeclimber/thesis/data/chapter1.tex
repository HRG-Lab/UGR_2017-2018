%%%%%%%%%%%%%%%%%%%%%%%%%%%%%%%%%%%%%%%%%%%%%%%%%%%
%
%  New template code for TAMU Theses and Dissertations starting Fall 2016.
%
%
%  Original Author: Sean Zachary Roberson
%  This version adapted for URS by Parasol lab.
%  Adapted from version 3.16.10, which was last updated on 9/29/2016.
%  URS adaptation last updated 1/9/2017.
%
%%%%%%%%%%%%%%%%%%%%%%%%%%%%%%%%%%%%%%%%%%%%%%%%%%%
%%%%%%%%%%%%%%%%%%%%%%%%%%%%%%%%%%%%%%%%%%%%%%%%%%%%%%%%%%%%%%%%%%%%%%
%%                           SECTION I
%%%%%%%%%%%%%%%%%%%%%%%%%%%%%%%%%%%%%%%%%%%%%%%%%%%%%%%%%%%%%%%%%%%%%


\pagestyle{plain} % No headers, just page numbers
%\pagenumbering{arabic} % Arabic numerals
%\setcounter{page}{1}

\chapter{INTRODUCTION AND LITERATURE REVIEW}

\section{Project Description}
Studying mobile RF data collection allows researchers and developers to examine
how mobile users receive and transmit RF data and can identify parts of the
process that can be improved. There are certain situations in which it is
necessary to gather this data manually. This is unfortunate because the gathering
of a significant amount of data can be very time consuming. For example, Texas
A\&M's Huff Research Group currently runs an experiment to generate a heat map of
received signal strength (RSS) by mounting a set of radios on a rover and
commanding it semi-autonomously to a number of locations. This process requires
one or two rover operators, and someone handling the RF measurements. This could
be greatly improved through the automation of the rovers and the integration of
Software Defined Radios (SDRs) and reconfigurable antennas. 

Software defined radio is the implementation of radio components, like mixers,
modulators, or filters in software rather than hardware. The technology was
developed in the 1990s, but it has become poised to be a much more disruptive
technology with the continuous improvement of supporting technologies \cite{Emerging_SDR}.
As a test platform, SDR grants the ability to monitor a number of frequencies
easily using a software front-end which can be automated. For instance, it would
be possible to develop a means of writing tests quickly with a scripting language
like Python to easily ensure repeatability of measurements. The development of
this automation could further feed forward into the development of antenna arrays
that are reconfigurable via autonomous movement of the platform. 

The field of robotics has also been booming in recent years with advances in
autonomous vehicles and general purpose robots like Rethink Robotics'
``Baxter''. There has been research in the control of robotic swarms
\cite{software_defined_radio} which could be leveraged to manipulate
reconfigurable antenna arrays. It is possible that a swarm of antennas deployed on
rovers or drones could be commanded to achieve a specific radiation pattern
based on their position and SDR reconfiguration. 

Finally, data visualization brings data to life and allows much greater
insight to what is gathered. There are a large amount of open-source libraries
for general purpose data visualization, e.g. Python's matplotlib library.
Leveraging this in a single consolidated cross-platform application would provide
a tool that lowers the cost (mostly in time) of studying spectral band coexistence.
Understanding these issues could serve to improve wireless connectivity in
autonomous vehicles and systems. 


\section{RF Measurement}

\section{Software Defined Radio (SDR)}

\section{Autonomous Drones}
