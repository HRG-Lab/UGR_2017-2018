%%%%%%%%%%%%%%%%%%%%%%%%%%%%%%%%%%%%%%%%%%%%%%%%%%%
%
%  New template code for TAMU Theses and Dissertations starting Fall 2016.
%
%
%  Original Author: Sean Zachary Roberson
%  This version adapted for URS by Parasol lab.
%  Adapted from version 3.16.10, which was last updated on 9/29/2016.
%  URS adaptation last updated 1/9/2017.
%
%%%%%%%%%%%%%%%%%%%%%%%%%%%%%%%%%%%%%%%%%%%%%%%%%%%
%%%%%%%%%%%%%%%%%%%%%%%%%%%%%%%%%%%%%%%%%%%%%%%%%%%%%%%%%%%%%%%%%%%%%
%%                           ABSTRACT
%%%%%%%%%%%%%%%%%%%%%%%%%%%%%%%%%%%%%%%%%%%%%%%%%%%%%%%%%%%%%%%%%%%%%

\chapter*{ABSTRACT}

\addcontentsline{toc}{chapter}{ABSTRACT} % Needs to be set to part, so the TOC doesnt add 'CHAPTER ' prefix in the TOC.

\pagestyle{plain} % No headers, just page numbers
\pagenumbering{arabic} % Arabic numerals
\setcounter{page}{1}
\begin{center}

\begin{singlespace}
\abstracttitle
\end{singlespace}
\vspace{2em}
\begin{singlespace}
\author \\
Department of \department \\
Texas A\&M University \\
\end{singlespace}
\vspace{2em}
\begin{singlespace}
Research Advisor: Dr. \ursadvisor \\
Department of \advisordepartment \\
Texas A\&M University \\
\end{singlespace}
\end{center}
\vspace{2em}

\indent The collection of large amounts of RF data in the field is often tedious
and narrowly focused. This work seeks to ease this process by investigating the
use of a number of technologies for the collection of this data. Specifically,
Software Defined Radio (SDR) has benefited from the development of enabling
technologies in recent years that have made it cheaper and more accesible. In
addition, autonomous drones have seen a meteoric rise in popularity. These two
technologies are combined to provide an autonomous means of investigating
multi-band and wide-band wireless networks improving both the volume and speed
of data collection.

\pagebreak{}
